In modern distributed systems, the slow speed of light is a true bottleneck. Due to this slow speed, packets sent over long distances will have significant latency. Depending on the context, this high latency can lead to poor performance of an application.


In this thesis we have investigated the Near-Far Computing Model to see if it is a viable solution to the latency problem as well as given an introduction to it. First we have found and analyzed some of what is already out there when it comes to Near-Far Computing. Then we have developed a high-level prototype in Emerald, and further used this prototype on PlanetLab to test Near-Far Computing architectures. Finally, we have found characteristics of each architecture.

We then compared these characteristics as well as discussing the results of the tests. Using this we have found if the Near-Far Computing Model is a viable solution, as well as describing what characterizes it. We have also given an introduction to Near-Far Computing, by using these characteristics and results.

Findings made in this thesis are confined to the environment used for testing, namely Emerald and PlanetLab.

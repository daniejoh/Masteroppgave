\chapter*{Abstract}
In modern distributed systems, the slow speed of light is a true bottleneck. Due to this slow speed, packets sent over long distances will have significant latency. Depending on the context, this high latency can lead to poor performance.


In this thesis, we have investigated The Near-Far Computing Model to see if it is a viable solution to the latency problem, as well as given an introduction to it. 

First, we have found and analyzed existing architectures that conform to The Near-Far Computing Model. Then we have developed a high-level prototype in Emerald and further used this prototype on PlanetLab to test two of these architectures. We have also found characteristics of each architecture. 

Using the data and characteristics we found, we have introduced The Near-Far Computing Model, and our finding indicates that it gives better results for latency-aware applications.

Findings made in this thesis are confined to the environment used for testing, namely Emerald and PlanetLab.
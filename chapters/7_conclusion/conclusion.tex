\chapter{Conclusion}

\section{Limitations}
Our experiments are limited to the Emerald and PlanetLab environment. This confines our measurements and results to these environments. Additionally, we acknowledge that simulating in the way we did, will not lead to real-life results. Therefore, these results and conclusions might not be applicable to other environments.


\section{Conclusion}
The slow speed of light is a bottleneck for computation and storage offloading in modern distributed computing. There is usually a long distance between the end-user and the big data centers. Due to the slow speed of light, packets that travel between them take much time. This has proven to be a problem for some contexts, and many have therefore sought to bring computation and storage to the edge. 

In this thesis, we have investigated and given an introduction to The Near-Far Computing Model as a solution to this problem. Near-Far Computing ensures that we have a node relatively close by the device that needs offloading. This ensures that they can offload workloads that are too heavy for their local device with minimal latency. 

We have created a high-level prototype in Emerald to test two of the architectures on PlanetLab. Our findings indicate that Near-Far Computing is a viable solution to mitigate the latency problem. 

We have introduced several architectures that are part of The Near-Far Computing spectrum and discussed their characteristics. Finally, we have discussed how these characteristics relate to The Near-Far Computing model and provided an introduction.

We have contributed to the research pool surrounding The Near-Far Computing Model, Emerald, and PlanetLab by doing analysis and experiments.



\section{Future Work}
\subsection{Application}
An application that is not a simulation should be developed to test Near-Far computing further to see if it is helpful in real-life contexts. An example of this could be an Augmented Reality application that relies on offloading. 

\subsection{Storage}
This thesis has not focused on testing storage offloading. The Near-Far Computing Model focuses on both computation and storage, and therefore research on storage offloading while using the model should be done.

\subsection{Environment}
Emerald and PlanetLab have provided an excellent environment for prototyping and simulating. However, to ensure that The Near-Far Computing Model works outside of this environment, future work should try to use other environments. An example of this is using actual mobile devices and not just simulations.

\subsection{Research}
Further research is needed on The Near-Far Computing Model. This thesis has only done high-level experimentation and found a small set of architectures that has been analyzed. More architectures and models should be visited and analyzed to see if they relate to The Near-Far Computing Model.


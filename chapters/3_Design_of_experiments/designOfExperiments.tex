\chapter{Design of experiments}\label{chapter:design_of_experiments}

This chapter will discuss how we will research and evaluate the architectures.


\section{Scope}
The goal of these analyses is to show how The Near-Far Computing Model relates to already existing architectures. The goal of the experiments is to show the viability of Near-Far Computing. First, we will research them and give an overview of them in the Architectures chapter. Then we will develop several of the architectures in Emerald and then conduct experiments on them. We will analyze and point out the characteristics of each architecture. Finally, we will discuss how the architectures are related to the Near-Far Computing Model.




\section{Simulation}
We will simulate the different architectures using Emerald and PlanetLab. The Emerald programming language will be used to make a high-level version of these architectures. Using the Emerald ecosystem comes with a bonus, namely that the Emerald VM is working as middleware for these experiments. This makes it easier to develop and show the implementation, as the code is more easily understood, and the architectures will be abstracted to something that is easily tested.








\section{Experiments}
The experiments done in this thesis will focus on how latency affects offloading. We will test with two different architectures and compare the results to see if Near-Far Computing is a viable solution for the latency problem. We will create a program that lets us easily change these parameters:
\begin{itemize}
    \item Latency
    \item Workload
    \item Processing power of the node
    \item Amount of interaction between the nodes
    \item Amount of nodes
\end{itemize}
We aim to make it easy to set up these tests with the help of Emeralds built in object mobility. 

\subsubsection{Offloading}
It is not always efficient to offload all the work. According to Mach and Becvar´s survey\cite{mach_mobile_2017}, we can divide offloading into three categories:
\begin{itemize}
    \item \textit{Local execution}, where all the work is done on the mobile device.
    \item \textit{Full offloading}, where we offload all work to server(s).
    \item \textit{Partial offloading}, where we offload some of the work, and do some local.
\end{itemize}
We will compare these three ways of offloading for both architectures.
Due to the high-level nature of our implementation, we argue that showing results for all the architectures are redundant. With this level of abstraction, some of the architectures will have very similar parameters. We therefore only do experiments on Cloudlets and Multi-Access Edge Computing.














\section{Theoretical analysis}
When evaluating, we will also have a theoretical part where we compare the characteristics of the different architectures. This is to find what characteristics are typical for Near-Far Computing.

\subsection{Transparency}
Since transparency is more about characteristics, it makes little sense to measure it quantitatively. We will therefore compare transparency characteristics from each architecture instead. We will take the different transparencies we discussed in section \ref{background:distributed_systems} and discuss how the different characteristics are related to these.




\section{How it relates to the Near-Far model}
When we have the characteristics and data laid out, we will discuss how it relates to The Near-Far Computing Model. We will compare all the data and characteristics to show what is best related.









\section{Summary}
In this chapter, we have discussed how we will conduct the experiments and analysis of the architectures. It will be an empirical part with tests and data and a theoretical part where we compare characteristics of the architectures.


%Notes:
% Hvis vi mangler info å skrive om kan vi skrive om storage?
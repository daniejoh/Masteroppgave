This chapter will discuss how we will research and evaluate the architectures.


\section{Scope}
Our goal of these experiments is to show how the Near-Far Computing Model relates to already existing architectures. First we will research them and give an overview of them in the Architectures chapter. Then we will develop several of the architectures in Emerald, and then conduct experiments on them. We will analyze and measure how the architectures perform and how they work. Finally we will discuss how the architectures related to the Near-Far Computing model.




\section{Simulation}
We will simulate the different architectures using Emerald and PlanetLab. The Emerald programming language will be used to make a high level version of these architectures. This comes with a bonus, namely that the Emerald VM is working as middleware for these experiments. This makes it easier to show experiments, as the code is easily understood and the architectures will be reduced to something that is easily tested.








\section{Experiments}
The experiments done in this thesis will focus on how latency affects offloading work. We will test with two different architectures and compare the results, to see if Near-Far computing is a viable solution for the latency problem. We will test with different amounts of latency for the Near node, while using a Far node that geographically very distant from, and therefore high latency from, Local or Near node. We will create a program that lets us easily change these parameters:
\begin{itemize}
    \item Latency
    \item Workload
    \item Processing power of the node
    \item Amount of interaction between the nodes
    \item Amount of nodes
\end{itemize}
We aim to make it easy to set up these tests with the help of Emeralds built in object mobility.

\subsubsection{Offloading}
It is not always efficient to offload all the work. According to Mach and Becvar´s survey\cite{mach_mobile_2017}, we can divide offloading into three categories:
\begin{itemize}
    \item \textit{Local execution}, where all the work is done on the mobile device.
    \item \textit{Full offloading}, where we offload all work to server(s).
    \item \textit{Partial offloading}, where we offload some of the work, and do some local.
\end{itemize}
We will compare these three ways of offloading for both architectures we test.
The rest of the architectures have very similar high level architectures, which is what we test with. We therefore assume that the results are very similar, and there is no need to present them in results.


\section{Theoretical analysis}
When evaluating we will also have a theoretical part where we compare the characteristics of the different architectures. This is to find what characteristics are typical for Near-Far computing.

\subsection{Transparency}
Since Distribution Transparency is more about key features, it makes little sense to measure it quantitatively. We will therefore compare transparency characteristics from each architecture instead. We will take the different transparencies we discussed in section \ref{background:distributed_systems} in the background, and lay out how the different characteristics relate to these.




\section{How it relates to the Near-Far model}
When we have the characteristics and data laid out we will be able to discuss how it relates to the Near-Far computing model. We will compare all the data and characteristics to show what is best related to the Near-Far computing model.









\section{Summary}
In this chapter we have discussed how we will conduct the experiments and analysis of the architectures. It will be an empirical part with tests and data, and a theoretical part where we compare characteristics of the architectures.


%Notes:
% Hvis vi mangler info å skrive om kan vi skrive om storage?